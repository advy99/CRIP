\documentclass[12pt, spanish]{article}
\usepackage[spanish]{babel}
\selectlanguage{spanish}
%\usepackage{natbib}
\usepackage{url}
\usepackage[utf8]{inputenc}
\usepackage{graphicx}
\graphicspath{{images/}}
\usepackage{parskip}
\usepackage{fancyhdr}
\usepackage{vmargin}
\usepackage{multirow}
\usepackage{float}
\usepackage{chngpage}
\usepackage{enumitem}
\usepackage{forloop}


\usepackage{amsfonts}
\usepackage{amsmath}

\usepackage{subcaption}

\usepackage{hyperref}
\usepackage[
    type={CC},
    modifier={by-nc-sa},
    version={4.0},
]{doclicense}

\hypersetup{
    colorlinks=true,
    linkcolor=blue,
    filecolor=magenta,
    urlcolor=cyan,
}

% para codigo
\usepackage{listings}
\usepackage{xcolor}


\usepackage[default]{sourcesanspro}

\setmarginsrb{2 cm}{1 cm}{2 cm}{2 cm}{1 cm}{1.5 cm}{1 cm}{1.5 cm}

\title{Práctica 3:\\
Funciones de un solo sentido.\hspace{0.05cm} }
\author{Antonio David Villegas Yeguas}
\date{\today}

\makeatletter
\let\thetitle\@title
\let\theauthor\@author
\let\thedate\@date
\makeatother

\pagestyle{fancy}
\fancyhf{}
\rhead{\theauthor}
\lhead{\thetitle}
\cfoot{\thepage}


\usepackage{caption}

\lstset{
language=C++,
basicstyle=\small\ttfamily,
numbers=left,
numbersep=5pt,
xleftmargin=20pt,
frame=tb,
framexleftmargin=20pt
}


\DeclareCaptionFormat{mylst}{\hrule#2#3}
\captionsetup[lstlisting]{format=mylst,labelfont=bf,singlelinecheck=off,labelsep=space}

\begin{document}

%%%%%%%%%%%%%%%%%%%%%%%%%%%%%%%%%%%%%%%%%%%%%%%%%%%%%%%%%%%%%%%%%%%%%%%%%%%%%%%%%%%%%%%%%

\begin{titlepage}
    \centering
    \vspace*{0.3 cm}
    \includegraphics[scale = 0.50]{ugr.png}\\[0.7 cm]
    %\textsc{\LARGE Universidad de Granada}\\[2.0 cm]
    \textsc{\large 4º CSI 2020/21}\\[0.5 cm]
    \textsc{\large Grado en Ingeniería Informática}\\[0.5 cm]
    \rule{\linewidth}{0.2 mm} \\[0.2 cm]
    { \huge \bfseries \thetitle}\\
    \rule{\linewidth}{0.2 mm} \\[1 cm]

    \begin{minipage}{0.4\textwidth}
        \begin{flushleft} \large
            \emph{Autor:}\\
            \theauthor\\
			 \emph{DNI:}\\
            77021623-M
            \end{flushleft}
            \end{minipage}~
            \begin{minipage}{0.4\textwidth}
            \begin{flushright} \large
            \emph{Asignatura: \\
            Criptografía}   \\
            \emph{Correo:}\\
            advy99@correo.ugr.es
        \end{flushright}
    \end{minipage}\\[0.5cm]

    {\large \thedate}\\[0.5cm]
    {\url{https://github.com/advy99/CRIP/}}
    {\doclicenseThis}

    \vfill

\end{titlepage}

%%%%%%%%%%%%%%%%%%%%%%%%%%%%%%%%%%%%%%%%%%%%%%%%%%%%%%%%%%%%%%%%%%%%%%%%%%%%%%%%%%%%%%%%%

\tableofcontents
\pagebreak

%%%%%%%%%%%%%%%%%%%%%%%%%%%%%%%%%%%%%%%%%%%%%%%%%%%%%%%%%%%%%%%%%%%%%%%%%%%%%%%%%%%%%%%%%


\section*{Introducción}

En esta práctica trabajaremos con funciones de un solo sentido de cara a comprender su uso para cifrar y firmar mensajes y hacer un resumen de mensajes, así como implementar algunas de las más utilizadas como la función de Rabin o RSA.


\section{Cifrado utilizando la función knapsack}

Este método consiste en generar una secuencia super-creciente (la suma de los terminos que preceden a un elemento es menor que dicho elemento). Con esta secuencia, para cifrar un mensaje en binario simplemente se escogen los elementos de la secuencia super-creciente en los que la posición asociada en el mensaje esté como verdadero.

Esto tiene la ventaja de que quien conozca la secuencia super-creciente puede resolver este problema utilizando un algoritmo voraz en muy pocos pasos, la longitud de la secuencia, sin embargo, si la secuencia no es supre-creciente este problema se trata de un problema NP completo, siendo imposible de resolver de forma práctica por un algoritmo conocido.

Sabiendo esto, el objetivo será que la secuencia super-creciente forme parte de la clave privada, de forma que quien la conozca pueda obtener los mensajes de forma rápida, y se utilice como clave pública una transformación a la secuencia super-creciente, de forma que escogeremos un $n$ mayor que la suma de la secuencia, y una transformación en $Z_n$ haga que la secuencia no sea super-creciente, convirtiéndolo en un problema NP completo.

Esto lo haremos escogiendo un $u$ tal que $gcd(n,u) = 1$ de cara a que $u$ tenga inverso en $Z_n$. Con esto, para un elemento de la secuencia super-creciente $a_i$, definiremos $a_i^* = u \cdot a_i mod n$, de forma que la secuencia $(a_1^*, \dots, a_k^*)$ será la clave pública y $((a_1, \dots, a_k),n, u)$ será la clave privada.

De esta forma, si alguien quiere mandarnos un mensaje solo tiene que cifrarlo con la clave pública:

$$
\sum_{i = 1}^{k} x_i a_i^*
$$


\begin{lstlisting}[caption={Algoritmo para descifrar un mensaje}]
INPUT: secuencia de bits (mensaje): m, clave puvlica: clave
OUTPUT: secuencia de bits que conforman el mensaje

resultado <- 0

Para cada elemento de m:
	resultado <- resultado + ( m[i] * clave[i])

Devolver resultado
\end{lstlisting}


Y para descifrar será necesario calcular el inverso de $u$ en $Z_n$, al número dado (mensaje cifrado), lo multiplicamos por el inverso de $u$, de forma que este número que se había calculado multiplicando por $u$, al multiplicar por su inverso obtenemos el número como si lo hubiéramos obtenido con la suma de elementos de la secuencia super-creciente, y no la secuencia transformada a $Z_n$.

De esta forma, como conocemos la secuencia super-creciente podemos obtener con un algoritmo voraz el mensaje original:

\begin{lstlisting}[caption={Algoritmo voraz para la función knapsack}]
INPUT: secuencia super-creciente: s, numero a descifrar: numero
OUTPUT: secuencia de bits que conforman el mensaje

resultado = [0,...,0] <- misma longitud que s

suma <- 0
i <- 0
Mientras suma != numero:
	Si suma + s[i] <= numero:
		resultado[i] <- 1
		suma <- suma + numero
	i <- i+1

Devolver resultado
\end{lstlisting}

La función para descifrar quedaría de la siguiente forma:

\begin{lstlisting}[caption={Algoritmo para descifrar un mensaje}]
INPUT: secuencia super-creciente: s, numero a descifrar: numero, n, u
OUTPUT: secuencia de bits que conforman el mensaje

inverso_u <- inverso(u, n) // inverso de u mod n
numero_a_descifrar <- (numero * inverso_u) mod n

Devolver algoritmo_voraz_knapsack(s, numero_a_descifrar)
\end{lstlisting}

Algunos ejemplos de la ejecución:

\begin{lstlisting}
-> ./bin/ejercicio1 13 1010100100011
Llave publica:
92627 32072 68013 67453 90667 56075 23392 2545 61496 26216
44414 16106 24194
Llave privada:
17 20 56 110 206 429 857 1700 3408 6810 13615 27239 54473
La llave privada generada es una secuencia super-creciente

N: 108943
U: 88758
Mensaje cifrado: 294152
Mensaje descifrado: 1010100100011
Mensaje       dado: 1010100100011


-> ./bin/ejercicio1 20 10100010001001010110
Llave publica:
12475022 4832564 10742065 12633531 5149582 8828615 6259145 6346381
6258445 13856235 12558758 2190142 3172143 11439258 5046114 1110425
2220850 4441700 8883400 3821229
Llave privada:
17 20 56 110 206 429 857 1700 3408 6810 13615 27239 54473 108944
217895 435799 871598 1743196 3486392 6972784
La llave privada generada es una secuencia super-creciente

N: 13945571
U: 11398085
Mensaje cifrado: 67909773
Mensaje descifrado: 10100010001001010110
Mensaje       dado: 10100010001001010110
\end{lstlisting}

Vemos como la clave privada es facil reconocer que es super-creciente, mientras que la pública parecen una secuencia de números sin correlación.

\section{Obteniendo y en ElGammal conociendo p, alfa y la imagen de x}

En este ejercicio se nos pide escoger un primo $p$ mayor que nuestro DNI, encontrar un elemento primitivo $\alpha$ en $Z_p$. Con esto, se nos da la función para obtener el valor $y$ de la clave pública de ElGammal, además de pedirnos usar nuestra fecha de nacimiento como número a calcular el inverso de la función.

Para resolver este ejercicio he escogido un primo $p$ que sea mayor que un número dado, y que $\frac{p}{2} - 1$ también sea primo. De esta forma, para calcular $\alpha$, simplemente he buscado un valor que el símbolo de Jacobi de dicho valor y $p$ sea igual a -1.

Con esto, sabiendo el valor de $\alpha$, el valor de la imagen y $p$, he utilizado el logaritmo discreto para obtener el inverso del número pedido:

\begin{lstlisting}[caption={Ejercicio 2}]
INPUT: posible p: posible_p, numero a calcular inverso: a_invertir
OUTPUT: inverso de a_invertir

p <- siguiente_primo(posible_p)

Mientras (p / 2) - 1 no sea primo:
	p <- siguiente_primo(p)

alpha <- 2

Mientras simbolo_jacobi(alpha, p) != -1:
	alpha <- alpha + 1

x <- logaritmo_discreto(alpha, a_invertir, p)

Devolver x
\end{lstlisting}

La ejecución con mi implementación:

\begin{lstlisting}
-> ./bin/ejercicio2 77021623 19990915
Utilizare el primo: 77022299
Un elemento primitivo de Z_77022299 es 2
El x tal que 2^x en Z_77022299 = 19990915 es 28098736
Comprobacion:
Fecha introducida: 19990915
Valor calculado con alpha ^ x: 19990915
\end{lstlisting}


\section{Obtener p y q dado un n y dos valores con la misma imagen para la función de Rabin}

En este ejercicio se nos da la función de Rabin: $f(x) = x^2$, también se nos proporciona el valor de $n = 48478872564493742276963$, y que $f (12) =144 = f (37659670402359614687722)$, y con esto tenemos que calcular los factores primos $p$ y $q$ de $n$.

Para resolver este ejercicio nos basaremos en que hemos encontrado dos números $x$ e $y$ con la misma función de Rabin, pero que no son iguales. Con esto sabemos que $x^2 \equiv y^2 mod n$, por lo que $(x + y)(x - y) = 0$, entonces sabemos que $n$ divide a $(x + y)(x - y)$, pero no a $(x - y)$, ya que $x$ e $y$ son distintos, luego $gcd(n, x - y)$ es un factor no trivial.

Para este ejercicio simplemente calculamos el máximo común divisor de $n$ con $x - y$ y de $n$ con $x - (n - y)$ para obtener $p$ y $q$:

\begin{lstlisting}
-> ./bin/ejercicio3 48478872564493742276963 12 37659670402359614687722
p = 159497098847
q = 303948303229
p * q         = 48478872564493742276963
n introducido = 48478872564493742276963
\end{lstlisting}


\section{Función resumen utilizando la construcción de Merkle-Damgar}

En este ejercicio se nos pide escoger dos cuadrados arbitrarios módulo $n$, $a_0$ y $a_1$, e implementar la función de compresión $h$:

$$
h(b,x) = x^2 a_0^b a_1^{1 - b}
$$

Donde $b$ es un bit, y $x$ el valor a comprimir. Como vemos, si el bit está a 1, se multiplica por $a_0$, mientras que si es 0, se multiplica por $a_1$, todo esto en $Z_n$.

Utilizaremos esta función con la construcción de Merkle-Damgar para implementar nuestra función resumen.

La construcción de Merkle-Damgar se basa en aplicar de forma anidada la función dada, utilizando como entrada para la siguiente iteración la salida de la iteración anterior. Por este motivo, para la primera iteración necesitaremos un vector (en nuestro caso valor) de inicialización, que será un parámetro más.

La función $h$ quedaría de la siguiente forma:


\begin{lstlisting}[caption={Funcion h}]
INPUT: n, x, b
OUTPUT: resultado de aplicar la funcion h

// valores escogidos al azar
a0 <- 4^2 mod n
a1 <- 23^2 mod n

resultado <- x^2 mod n

Si b:
	resultado <- resultado * a0
En otro caso:
	resultado <- resultado * a1

Devolver resultado
\end{lstlisting}

Y la función resumen sería la siguiente:

\begin{lstlisting}[caption={Funcion resumen}]
INPUT: n, x_inicial, mensaje (secuencia de bits)
OUTPUT: resultado de aplicar la funcion resumen

resultado <- funcion_h(n, x_inicial, mensaje[0])

Para i = 1 hasta la longitud de mensaje:
	resultado <- funcion_h(n, resultado, mensaje[i])
	i <- i + 1

Devolver resultado
\end{lstlisting}

Como vemos, para cada iteración utilizamos el resultado anterior, haciendo la construcción de Merkle-Damgar.


\begin{lstlisting}
-> ./bin/ejercicio4  1231241123 2 001010
N no es primo, utilizando el siguiente primo.
N = 1231241147
Resultado: 998023034

-> ./bin/ejercicio4  1231241123 2 001011
N no es primo, utilizando el siguiente primo.
N = 1231241147
Resultado: 246642335
\end{lstlisting}

Estos son algunos ejemplos, donde vemos como cambiando el mensaje solo al final y en un bit, sigue modificando totalmente el resultado, como esperábamos de la función resumen.

\section{Descifrar RSA conociendo p y q}

En este ejercicio se nos pide que utilicemos como $p$ el siguiente primo de nuestro número de identidad, y $q$ el siguiente primo a nuestra fecha de nacimiento. De esfa forma, obteniendo $p$ y $q$, seleccionar un $e$ tal que $gcd(e, \phi(n)) = 1$, donde $\phi(n) = (p - 1)(q - 1)$. Con esto, se nos da la función RSA:

$$
f: Z_n \rightarrow Z_n, x \rightarrow x^e
$$

Y se nos pide calcular el inverso de $1234567890$.

Como vemos, en este caso conocemos la clave pública ($n$ y $e$), y la clave privada $d$ (podemos calcular el inverso de $e$ módulo $\phi(n)$). Esto hace que para este ejercicio solo tengamos que descifrar un mensaje, es decir, elevarlo a $d$, y como siempre, módulo $n$.

\begin{lstlisting}[caption={Funcion desencriptar RSA}]
INPUT: p, q, numero_descifrar
OUTPUT: numero descifrado


phi_n <- (p - 1)(q - 1)
e <- 2

Mientras gcd(e, phi_n) != 1:
	e <- e + 1

d <- inverso(e, phi_n)

resultado <- numero_descifrar^d mod (p * q)

Devolver resultado
\end{lstlisting}

Y con esto ya tenemos el resultado:

\begin{lstlisting}
-> ./bin/ejercicio5 77021623 19990915 1234567890
p = 77021647
q = 19990931
n = 1539734430683357
phi_n = 1539734333670780
e = 7
d = 659886143001763
El mensaje es : 488483345273870
Comprobacion:
Mensaje cifrado con resultado^e (ciframos con RSA el mensaje) = 1234567890
Mensaje cifrado original: 1234567890
\end{lstlisting}


\section{Obtener p y q en RSA conociendo la clave publica y privada}

En este caso se nos da $n$, $e$ y $d$, y se nos pide calcular $p$ y $q$.

Para poder calcular $p$ y $q$ tenemos que conocer $\phi(n)$, que en este caso es conocido, ya que sabemos que $d$ es el inverso de $e$ módulo $\phi(n)$, luego $\phi(n) = e \cdot d - 1$.

Con esto, podemos descomponer $\phi(n)$ como $2^a \cdot b$, y escoger un número $0 < x < n$ de forma aleatoria, si $gcd(x, n) \not = 1$ ya hemos encontrado un factor, y podemos parar. Si vale 1 podemos definir $y = x^b mod n$, si $y = \pm 1 (mod n)$ el algoritmo ha fallado, ya que la idea es buscar un $y$ tal que $y^2 = \pm 1 (mod n)$, por lo que no podemos partir de un $1$ o un $-1$. Si llegamos a un punto en el que $y^2 = 1 (mod n)$, hemos encontrado un $y \not = 1$ pero $y^2 = 1$, entonces hemos encontrado un factor no trivial, con lo que obtenemos los factores primos de $n$ como $gcd(n, y + 1)$ y $gcd(n, y - 1)$, si $y^2 = -1 (mod n)$ el algoritmo ha fallado.


\begin{lstlisting}[caption={Funcion descomponer n}]
INPUT: n, d, e
OUTPUT: p, q

phi_n <- (e * d - 1) mod n

b <- phi_n
a <- 0

Mientras b sea par:
	b <- b / 2
	a <- a + 1

x <- valor aleatorio entre 0 y n

y <- x^b mod n

p <- -1
q <- -1

Si y != 1 Y y != -1:
	Hacer:
		// para tener una copia del y anterior
		// que es el que interesa
		z <- y
		y <- y^2 mod n
	Mientras y != 1 Y y != -1

	Si y = 1:
		p <- gcd(n, z + 1)
		q <- gcd(n, z - 1)

Devolver p, q
\end{lstlisting}

Y con esto ya tenemos los factores de $n$:

\begin{lstlisting}
-> ./bin/ejercicio6 50000000385000000551 5 10000000074000000101
p = 5000000029
q = 10000000019
50000000385000000551 = 5000000029 * 10000000019
Comprobacion:
n   = 50000000385000000551
p*q = 50000000385000000551
\end{lstlisting}


\subsection{Similitudes con el algoritmo de Miller-Rabin y el ejercicio 3}

El algoritmo propuesto tiene grandes similitudes con el algoritmo de Miller-Rabin y la solución propuesta en el ejercicio 3.

Con respecto al algoritmo de Miller-Rabin vemos como ambos se basan en la idea de buscar un elemento distinto de $1$ cuyo cuadrado sea $1$, para de esta forma encontrar un factor.

También podemos ver como en el ejercicio 3, donde teníamos una $x$ y $y$  tal que $(x + y)(x - y)$ dividía a $n$, luego podiamos obtener $p$ y $q$, este ejercicio se basa en lo mismo, en nuestro caso $x$ sería el valor calculado por el algoritmo cuyo cuadrado vale 0, e $y = 1$.

Estos tres ejercicios se basan en la misma idea de buscar elementos cuyos cuadrados valgan 1, pero que los dichos elementos no sean 1, encontrando así un factor.


\section{Implementando la firma RSA}

En este ejercicio implementaremos un sistema de firma y verificación de firma digital utilizando RSA.

Este ejercicio se basa en tres partes:

\begin{enumerate}
	\item Generación de llaves.
	\item Firma de un fichero.
	\item Verificación de la firma asociada a un fichero.
\end{enumerate}

\subsection{Generación de llaves}

RSA se basa en seleccionar dos primos grandes y distantes entre si, y calcular $n$ como el entero de Blum de esos dos primos.

Una vez tenemos $n$, calcularemos $e$ como un primo relativo con $\phi(n)$, por lo que tendrá inverso módulo $\phi(n)$, que será nuestro $d$.

De esta forma, la llave pública será la pareja $(n, e)$, mientras que la llave privada será $d$.

\subsection{Generación de la firma asociada a un fichero}

Una vez tenemos la llave pública y privada, para firmar un fichero necesitaremos la clave privada.

Con RSA para firmar un mensaje basta con calcular el mensaje elevado a $d$ módulo $n$, de esta forma solo la persona con la clave privada puede firmar un mensaje.

Normalmente no se firma el archivo, si no un resumen de este. En nuestro caso utilizaremos SHA1 para obtener el resumen, y firmaremos dicho resumen. Como SHA1 nos devuelve una salida en hexadecimal, simplemente pasaremos esa salida a decimal para poder operar.

\begin{lstlisting}[caption={Firmar mensaje RSA}]
INPUT: mensaje, d, n
OUTPUT: firma del mensaje f

resumen_mensaje <- SHA1(mensaje)
f <- resumen_mensaje^d mod n

Devolver f
\end{lstlisting}


\subsection{Verificación de la firma asociada a un fichero}

Para verificar la firma, como $d$, la clave privada, es el inverso de $e$ módulo $\phi(n)$, simplemente basta con comprobar que la firma elevado a $e$ es el mensaje original módulo $n$.

\begin{lstlisting}[caption={Verificar mensaje RSA}]
INPUT: mensaje, firma, e, n
OUTPUT: booleano: verdadero si mensaje se ha firmado con firma

resumen_mensaje <- SHA1(mensaje)
comprobacion <- firma^e mod n

Devolver comprobacion == (resumen mod n)
\end{lstlisting}


Para probar el ejercicio he generado las llaves, firmado un fichero y verificado la firma para comprobar que es correcto. También he probado a modificar levemente el fichero (introducir un espacio, o cambio pequeño) y comprobar que la firma original ya no es válida, ya que se ha modificado el fichero.

\begin{lstlisting}
CRIP/practica3 on main [!?]
-> ./bin/ejercicio7_generar_llaves llave.pub llave

CRIP/practica3 on main [!?]
-> cat llave.pub
10962077638758118016385560362123
3

CRIP/practica3 on main [!?]
-> cat llave
7308051759172073995266668669659

CRIP/practica3 on main [!?]
-> ./bin/ejercicio7_firmar_archivo Makefile llave.pub llave firma

CRIP/practica3 on main [!?]
-> ./bin/ejercicio7_verificar_firma Makefile firma llave.pub
El mensaje almacenado en Makefile corresponde con la firma
almacenada en firma

CRIP/practica3 on main [!?]
-> echo "" >> Makefile # Inserto salto de linea al final

CRIP/practica3 on main [!?]
-> ./bin/ejercicio7_verificar_firma Makefile firma llave.pub
AVISO: El mensaje almacenado en Makefile NO corresponde con
la firma almacenada en firma
\end{lstlisting}

\end{document}
