\documentclass[12pt, spanish]{article}
\usepackage[spanish]{babel}
\selectlanguage{spanish}
%\usepackage{natbib}
\usepackage{url}
\usepackage[utf8x]{inputenc}
\usepackage{graphicx}
\graphicspath{{images/}}
\usepackage{parskip}
\usepackage{fancyhdr}
\usepackage{vmargin}
\usepackage{multirow}
\usepackage{float}
\usepackage{chngpage}
\usepackage{enumitem}
\usepackage{forloop}


\usepackage{amsfonts}

\usepackage{subcaption}

\usepackage{hyperref}
\usepackage[
    type={CC},
    modifier={by-nc-sa},
    version={4.0},
]{doclicense}

\hypersetup{
    colorlinks=true,
    linkcolor=blue,
    filecolor=magenta,
    urlcolor=cyan,
}

% para codigo
\usepackage{listings}
\usepackage{xcolor}


\usepackage[default]{sourcesanspro}

\setmarginsrb{2 cm}{1 cm}{2 cm}{2 cm}{1 cm}{1.5 cm}{1 cm}{1.5 cm}

\title{Práctica 1:\\
Aritmética modular.\hspace{0.05cm} }
\author{Antonio David Villegas Yeguas}
\date{\today}

\renewcommand*\contentsname{hola}

\makeatletter
\let\thetitle\@title
\let\theauthor\@author
\let\thedate\@date
\makeatother

\pagestyle{fancy}
\fancyhf{}
\rhead{\theauthor}
\lhead{\thetitle}
\cfoot{\thepage}


\usepackage{caption}

\lstset{
language=C++,
basicstyle=\small\ttfamily,
numbers=left,
numbersep=5pt,
xleftmargin=20pt,
frame=tb,
framexleftmargin=20pt
}

\renewcommand*\thelstnumber{\arabic{lstnumber}:}

\DeclareCaptionFormat{mylst}{\hrule#1#2#3}
\captionsetup[lstlisting]{format=mylst,labelfont=bf,singlelinecheck=off,labelsep=space}

\begin{document}

%%%%%%%%%%%%%%%%%%%%%%%%%%%%%%%%%%%%%%%%%%%%%%%%%%%%%%%%%%%%%%%%%%%%%%%%%%%%%%%%%%%%%%%%%

\begin{titlepage}
    \centering
    \vspace*{0.3 cm}
    \includegraphics[scale = 0.50]{ugr.png}\\[0.7 cm]
    %\textsc{\LARGE Universidad de Granada}\\[2.0 cm]
    \textsc{\large 4º CSI 2020/21}\\[0.5 cm]
    \textsc{\large Grado en Ingeniería Informática}\\[0.5 cm]
    \rule{\linewidth}{0.2 mm} \\[0.2 cm]
    { \huge \bfseries \thetitle}\\
    \rule{\linewidth}{0.2 mm} \\[1 cm]

    \begin{minipage}{0.4\textwidth}
        \begin{flushleft} \large
            \emph{Autor:}\\
            \theauthor\\
			 \emph{DNI:}\\
            77021623-M
            \end{flushleft}
            \end{minipage}~
            \begin{minipage}{0.4\textwidth}
            \begin{flushright} \large
            \emph{Asignatura: \\
            Criptografía}   \\
            \emph{Correo:}\\
            advy99@correo.ugr.es
        \end{flushright}
    \end{minipage}\\[0.5cm]

    {\large \thedate}\\[0.5cm]
    {\url{https://github.com/advy99/CRIP/}}
    {\doclicenseThis}

    \vfill

\end{titlepage}

%%%%%%%%%%%%%%%%%%%%%%%%%%%%%%%%%%%%%%%%%%%%%%%%%%%%%%%%%%%%%%%%%%%%%%%%%%%%%%%%%%%%%%%%%

\tableofcontents
\pagebreak

%%%%%%%%%%%%%%%%%%%%%%%%%%%%%%%%%%%%%%%%%%%%%%%%%%%%%%%%%%%%%%%%%%%%%%%%%%%%%%%%%%%%%%%%%


\section*{Introducción}

En esta práctica repasaremos conceptos de las asignaturas Álgebra Lineal y Métodos Discretos y Lógica y Métodos Discretos de cara a implementar algoritmos que nos permitan trabajar en anillos y cuerpos de $\mathbb{Z}$.

\section{Algoritmo extendido de Euclides}

En este primer ejercicio se nos pide implementar el algoritmo extendido de Euclides. Este algoritmo nos permitirá obtener el máximo común divisor de dos números que recibirá como entrada.

Este algoritmo se basa en ir realizando divisiones sucesivas hasta obtener un cero, y el último resto distinto de cero será el máximo común divisor. Además, también obtendremos un $s$ y un $t$, tal que si buscamos el máximo común divisor de $a$ y $b$, $mcd(a, b) = a \cdot s + b \cdot t$.

La implementación es la siguiente:

\begin{lstlisting}[caption={Algoritmo extendido de Euclides}]
INPUT: a, b
OUTPUT: r <- mcd(a,b), s, t

Si a == 0 o b == 0
	Devolver vacio

r0 <- a
r1 <- b
s0 <- 1
s1 <- 0
t0 <- 0
t1 <- 1

Mientras r1 != 0:
	q <- r0 / r1
	r0, r1 <- r1, r0 - r1 * q
	s0, s1 <- s1, s0 - s1 * q
	t0, t1 <- t1, t0 - t1 * q

Devolver r0, s0, t0
\end{lstlisting}

Como vemos, el algoritmo se basa en ir dividiendo $a$ por $b$, y actualizar sus valores, $a$ pasa a valer $b$, y $b$ el residuo de la división. De esta forma, cuando $b$ llegue a cero, $a$ valdrá el residuo anterior, que será el máximo común divisor.

En $s$ y $t$ iremos almacenando los valores de la identidad de Bezout de cara a tener los coeficientes con los que se puede expresar el máximo común divisor.


\section{Inverso modular de dos primos relativos}

Para este ejercicio vamos a calcular el inverso de un número $a$ módulo $b$, siendo $a$ y $b$ dos primos relativos.


\section{Potencia modular}

\section{Test de Miller-Rabin}

\section{Logaritmo discreto}

\section{Raices cuadradas modulares}

\subsection{Raiz cuadrada modular}

\subsection{Obtener todas las raices}


\section{Métodos de factorización}

\subsection{Factorización utilizando el método de Fermat}

\subsection{Factorización utilizando el algoritmo de $\rho$ de Pollard}



\section{Comparación de tiempos}


% \begin{thebibliography}{9}
%
%
% \end{thebibliography}

\end{document}
