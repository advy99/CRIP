\documentclass[12pt, spanish]{article}
\usepackage[spanish]{babel}
\selectlanguage{spanish}
%\usepackage{natbib}
\usepackage{url}
\usepackage[utf8x]{inputenc}
\usepackage{graphicx}
\graphicspath{{images/}}
\usepackage{parskip}
\usepackage{fancyhdr}
\usepackage{vmargin}
\usepackage{multirow}
\usepackage{float}
\usepackage{chngpage}
\usepackage{enumitem}
\usepackage{forloop}


\usepackage{amsfonts}
\usepackage{amsmath}

\usepackage{subcaption}

\usepackage{hyperref}
\usepackage[
    type={CC},
    modifier={by-nc-sa},
    version={4.0},
]{doclicense}

\hypersetup{
    colorlinks=true,
    linkcolor=blue,
    filecolor=magenta,
    urlcolor=cyan,
}

% para codigo
\usepackage{listings}
\usepackage{xcolor}


\usepackage[default]{sourcesanspro}

\setmarginsrb{2 cm}{1 cm}{2 cm}{2 cm}{1 cm}{1.5 cm}{1 cm}{1.5 cm}

\title{Práctica 1:\\
Aritmética modular.\hspace{0.05cm} }
\author{Antonio David Villegas Yeguas}
\date{\today}

\makeatletter
\let\thetitle\@title
\let\theauthor\@author
\let\thedate\@date
\makeatother

\pagestyle{fancy}
\fancyhf{}
\rhead{\theauthor}
\lhead{\thetitle}
\cfoot{\thepage}


\usepackage{caption}

\lstset{
language=C++,
basicstyle=\small\ttfamily,
numbers=left,
numbersep=5pt,
xleftmargin=20pt,
frame=tb,
framexleftmargin=20pt
}


\DeclareCaptionFormat{mylst}{\hrule#2#3}
\captionsetup[lstlisting]{format=mylst,labelfont=bf,singlelinecheck=off,labelsep=space}

\begin{document}

%%%%%%%%%%%%%%%%%%%%%%%%%%%%%%%%%%%%%%%%%%%%%%%%%%%%%%%%%%%%%%%%%%%%%%%%%%%%%%%%%%%%%%%%%

\begin{titlepage}
    \centering
    \vspace*{0.3 cm}
    \includegraphics[scale = 0.50]{ugr.png}\\[0.7 cm]
    %\textsc{\LARGE Universidad de Granada}\\[2.0 cm]
    \textsc{\large 4º CSI 2020/21}\\[0.5 cm]
    \textsc{\large Grado en Ingeniería Informática}\\[0.5 cm]
    \rule{\linewidth}{0.2 mm} \\[0.2 cm]
    { \huge \bfseries \thetitle}\\
    \rule{\linewidth}{0.2 mm} \\[1 cm]

    \begin{minipage}{0.4\textwidth}
        \begin{flushleft} \large
            \emph{Autor:}\\
            \theauthor\\
			 \emph{DNI:}\\
            77021623-M
            \end{flushleft}
            \end{minipage}~
            \begin{minipage}{0.4\textwidth}
            \begin{flushright} \large
            \emph{Asignatura: \\
            Criptografía}   \\
            \emph{Correo:}\\
            advy99@correo.ugr.es
        \end{flushright}
    \end{minipage}\\[0.5cm]

    {\large \thedate}\\[0.5cm]
    {\url{https://github.com/advy99/CRIP/}}
    {\doclicenseThis}

    \vfill

\end{titlepage}

%%%%%%%%%%%%%%%%%%%%%%%%%%%%%%%%%%%%%%%%%%%%%%%%%%%%%%%%%%%%%%%%%%%%%%%%%%%%%%%%%%%%%%%%%

\tableofcontents
\pagebreak

%%%%%%%%%%%%%%%%%%%%%%%%%%%%%%%%%%%%%%%%%%%%%%%%%%%%%%%%%%%%%%%%%%%%%%%%%%%%%%%%%%%%%%%%%


\section*{Introducción}

En esta práctica repasaremos conceptos de las asignaturas Álgebra Lineal y Métodos Discretos y Lógica y Métodos Discretos de cara a implementar algoritmos que nos permitan trabajar en anillos y cuerpos de $\mathbb{Z}$.

\section{Algoritmo extendido de Euclides}

En este primer ejercicio se nos pide implementar el algoritmo extendido de Euclides. Este algoritmo nos permitirá obtener el máximo común divisor de dos números que recibirá como entrada.

Este algoritmo se basa en ir realizando divisiones sucesivas hasta obtener un cero, y el último resto distinto de cero será el máximo común divisor. Además, también obtendremos un $s$ y un $t$, tal que si buscamos el máximo común divisor de $a$ y $b$, $mcd(a, b) = a \cdot s + b \cdot t$.

La implementación es la siguiente:

\begin{lstlisting}[caption={Algoritmo extendido de Euclides}]
INPUT: a, b
OUTPUT: r <- mcd(a,b), s, t

Si a == 0 o b == 0
	Devolver vacio

r0 <- a
r1 <- b
s0 <- 1
s1 <- 0
t0 <- 0
t1 <- 1

Mientras r1 != 0:
	q <- r0 / r1
	r0, r1 <- r1, r0 - r1 * q
	s0, s1 <- s1, s0 - s1 * q
	t0, t1 <- t1, t0 - t1 * q

Devolver r0, s0, t0
\end{lstlisting}

Como vemos, el algoritmo se basa en ir dividiendo $a$ por $b$, y actualizar sus valores, $a$ pasa a valer $b$, y $b$ el residuo de la división. De esta forma, cuando $b$ llegue a cero, $a$ valdrá el residuo anterior, que será el máximo común divisor.

En $s$ y $t$ iremos almacenando los valores de la identidad de Bezout de cara a tener los coeficientes con los que se puede expresar el máximo común divisor.

Ejemplo de ejecución:

\begin{lstlisting}
> ./bin/ejercicio1 10 6
El maximo comun divisor de 10 y 6 es 2
u: -1
v: 2
Comprobacion usando boost: 2

> ./bin/ejercicio1 123 34235
El maximo comun divisor de 123 y 34235 es 41
u: -278
v: 1
Comprobacion usando boost: 41

> ./bin/ejercicio1 1134513413151345345423 331456134616113464235
El maximo comun divisor de
	1134513413151345345423 y 331456134616113464235 es 3
u: -14539121067186444509
v: 49764738508334856546
Comprobacion usando boost: 3
\end{lstlisting}


\section{Inverso modular de dos primos relativos}

Para este ejercicio vamos a calcular el inverso de un número $a$ módulo $b$, siendo $a$ y $b$ dos primos relativos.

Si $a$ y $b$ son primos resativos, su máximo común divisor vale uno, por lo que sabemos por la identidad de Bezout que:

$$ mcd(a, b) = 1 = u \cdot a + v \cdot b $$

Y también sabemos que estamos trabajando en los restos de dividir por $b$, es decir $\mathbb{Z}_b$, luego:

$$1 = u \cdot a + v \cdot b $$
$$1 = u \cdot a $$
$$a^{-1} = u$$

Ya que en $\mathbb{Z}_b$ el valor de $b$ es cero. De esta forma, con el algoritmo extendido de Euclides y sabiendo que $a$ y $b$ son primos relativos podemos obtener el inverso de $a$ módulo $b$.

En este caso utilizamos el algoritmo del ejercicio anterior, usando el segundo valor devuelto, el de $s$. También podríamos volverlo a escribir sin calcular $t$, ya que no es necesario para el cálculo.


Ejemplo de ejecución:

\begin{lstlisting}
> ./bin/ejercicio2 4123124122 15221235235
El inverso de 4123124122 modulo 15221235235 es 8391178143
Comprobacion usando boost: 8391178143

> ./bin/ejercicio2 4123124122 152212352352351
El inverso de 4123124122 modulo 152212352352351 es 88431670057630
Comprobacion usando boost: 88431670057630

> ./bin/ejercicio2 4123124112422 152212352352351
El inverso de 4123124112422 modulo 152212352352351 es 120410964876188

Comprobacion usando boost: 120410964876188
\end{lstlisting}

\section{Potencia modular}

Ahora se nos pide implementar un algoritmo de forma que calculemos $a^b \mod n$ para cualquier $a$, $b$ y $n$ enteros positivos.

Este ejercicio parece simple a primera vista, pero el problema es que si se tratan de valores excesivamente grandes la operación normal de exponenciación no es suficiente, ya que desbordaría los registros y memoria de la máquina.

Para evitar esto, sabiendo que vamos a trabajar en $\mathbb{Z}_{n}$, la idea es aprovechar la representación binaria del exponente para saber cuando tenemos que multiplicar el resultado por la base, y al utilizar representación binaria, cada vez que desplacemos un bit del exponente, elevar al cuadrado la base, de forma que la operación sea equivalente.

\newpage

\begin{lstlisting}[caption={Algoritmo de exponenciación modular}]
INPUT: base, exponente, modulo
OUTPUT: resultado <- (base^exponente) mod modulo

resultado <- 1

Mientras exponente > 0:
	Si el bit menos significativo del exponente == 1:
		resultado <- (resultado * base) mod modulo

	exponente <- exponente desplazando su valor un bit a la derecha
	base <- (base * base) mod modulo

Devolver resultado
\end{lstlisting}


Ejemplo de ejecución:

\begin{lstlisting}
> ./bin/ejercicio3 8324234234234234975289357802394578
	348934242527545742545720573489025 234568245760248764

8324234234234234975289357802394578 elevado a
	348934242527545742545720573489025 modulo 234568245760248764
	es 205725323563277504
Comprobacion (utilizando powm): 8324234234234234975289357802394578
	elevado a 348934242527545742545720573489025 modulo
	234568245760248764 es 205725323563277504

> ./bin/ejercicio3 8324234234 9025 23456824576064
8324234234 elevado a 9025 modulo 23456824576064 es 689449604480
Comprobacion (utilizando powm): 8324234234 elevado a
	9025 modulo 23456824576064 es 689449604480

> ./bin/ejercicio3 8324234234 9025 23
8324234234 elevado a 9025 modulo 23 es 4

Comprobacion (utilizando powm): 8324234234 elevado a 9025 modulo 23 es 4
\end{lstlisting}

\section{Test de Miller-Rabin}

El test de Miller-Rabin se trata de un test de primalidad con una tasa de acierto del 75\% si el resultado es que el número es primo y determinista si el número no es primo. Además cada ejecución es independiente por lo que podemos lanzar el test multiples veces y si las pasa todas, podemos asegurar con una probabilidad muy alta que un número es primo.

Este test se basa en dos ideas:

\begin{enumerate}
	\item Si $p$ es un número primo, $\mathbb{Z}_p$ es un cuerpo, por lo que $x^2 - 1 = 0$ tiene solo dos raices (test de Fermat).
	\item Si para un $a$ tal que $a \neq 0$ entonces $a^{-1} = 1$ en $\mathbb{Z}_p$.
\end{enumerate}

Si $p$ es par y distinto de dos, rápidamente podemos responder que no es primo.

Este test no funciona para primos por debajo de cinco, aunque no supone un problema ya que podemos tratar ese caso particular.

El test se basa en descomponer $p - 1$ como $2^u \cdot s$ con $s$ impar, escoger una semilla aleatoria $a$ tal que $1 < a < p - 1$ y con esa semilla obtener la siguiente lista $L$:

$$L = {a^s, a^{2 \cdot s}, a^{2^2 \cdot s}, ..., a^{2^{u - 1} \cdot s}, a^{2^{u} \cdot s}}$$

De esta forma, podrían darse los siguientes casos:

\begin{enumerate}
	\item Que la primera potencia $a^s$ valga $1$ o $p - 1$, entonces $p$ \textbf{podría ser primo}.
	\item Que ningun elemento de $L$ sea uno, luego no superaría el test de Fermat, y $p$ \textbf{no es primo}.
	\item Que encontremos una potencia que valga uno, sin que esté precedida de una potencia que valga menos uno. En este caso $p$ \textbf{no es primo}, ya que hemos encontrado una raiz que no es ni uno ni menos uno en $\mathbb{Z}_p$.
	\item Que encontremos una potencia que valga menos uno, luego la siguiente valdrá uno, y por lo tanto $p$ \textbf{podría ser primo}.
\end{enumerate}

Como vemos, en cualquier momento que encontremos un uno o menos uno podemos parar el algoritmo, y no es necesario calcular la lista completa. Otro detalle a tener en cuenta en la implementación es que no es necesario calcular cada valor independiente de la lista, un valor se obtiene como el cuadrado del anterior.

\begin{lstlisting}[caption={Test de Miller-Rabin}]
INPUT: p
OUTPUT: verdadero si p podria ser primo, falso si no lo es

resultado <- falso

Si p <= 5:
	resultado <- numero == 2 o numero == 3 o numero == 5
si no:
	Si p es par:
		resultado <- falso
	Si no:
		Descomponemos p - 1 como 2^u * s (almacenamos u y s)
		a <- aleatorio entre 2 y p - 1
		a <- a^s mod p

		resultado <- a == 1 o a == p - 1
		continuar <- !resultado

		Para i = 1 hasta i < u Y continuar:
			a <- a^2 mod p

			Si a == p - 1:
				resultado <- verdadero
				continuar <- falso

			Si a == 1
				resultado <- falso
				continuar <- falso

			i <- i + 1

Devolver resultado
\end{lstlisting}

Como he comentado, este test nos dice si $p$ es primo un 75\% de probabilidad de acertar en caso positivo. Como cada ejecución es independiente y si responde que no es primo, sabemos con certeza que no lo es, podemos ejecutar diez veces el test para comprobar si es primo, lo que nos daría una probabilidad de acierto del $99'9999046\%$.

De esta forma, nuestra función de si un número es primo sería la siguiente:

\begin{lstlisting}[caption={Función para comprobar si un numero es primo}]
INPUT: p
OUTPUT: verdadero si p es un pseudoprimo, falso si no lo es

resultado <- verdadero

i <- 0

Mientras i < 10 Y resultado:
	resultado = resultado Y test_miller_rabin(p)
	i <- i +1

Devolver resultado
\end{lstlisting}

También podemos tener una función que nos diga el siguiente primo dado un número, comprobando si el siguiente impar es primo, y si no lo es, sumando dos. Si el número inicial es par, le restamos uno para que al sumar dos nos de el siguiente impar.

\begin{lstlisting}[caption={Función para encontrar el siguiente primo}]
INPUT: p
OUTPUT: resultado <- siguiente primo a p

resultado <- p

Si p es par:
	resultado <- resultado - 1

Hacer:
	resultado <- resultado + 2
Mientras !es_primo(resultado)

Devolver resultado
\end{lstlisting}

Ejemplo de ejecución:

\begin{lstlisting}
> ./bin/ejercicio4 234234235235324534563463456345345314531463157463
234234235235324534563463456345345314531463157463 es primo
El siguiente primo es 234234235235324534563463456345345314531463157821
Comprobacion usando boost:
234234235235324534563463456345345314531463157463 es primo

> ./bin/ejercicio4 1452312356134612363421
1452312356134612363421 no es primo

El siguiente primo es 1452312356134612363481
Comprobacion usando boost:
1452312356134612363421 no es primo

> ./bin/ejercicio4 36780481
36780481 es primo
El siguiente primo es 36780487
Comprobacion usando boost:
36780481 es primo
\end{lstlisting}


\section{Logaritmo discreto}

En este caso el objetivo es calcular un $b$ tal que $a^b = c$ en $\mathbb{Z}_n$, conociendo $a$, $c$ y $n$.

Al principio podríamos pensar que simplemente podríamos probar con diferentes exponentes, hasta dar con el resultado, ya que el número de elementos de $\mathbb{Z}_n$ es finito, sin embargo esto sería muy lento para un $n$ demasiado grande, y este algoritmo estaría acotado superiormente por $\phi(n)$.

Por este motivo Daniel Shanks sugirió el algoritmo de paso enano paso gigante, que será el que implementemos. Este algoritmo solo funciona si $n$ es primo, ya que es necesario que $\mathbb{Z}_n$ sea cíclico, por lo que nos centraremos en los casos de módulos primos, así que a partir de ahora $n$ pasaré a llamarlo $p$.

Antes de explicar el algoritmo, hay que tener en cuenta algunos detalles:

\begin{enumerate}
	\item Por el Teorema de Fermat sabemos que $a^{p - 1} = 1$, luego en caso de encontrar el exponente $b$ sabemos que estará en el intervalo $1 \leq b \leq p - 1$.
	\item  También sabemos que dado un número natural $s$, cualquier $n$ entre uno y $s^2$ lo podemos expresar como $n = t \cdot s - r$ con $0 \leq r \leq s - 1$ y $1 \leq t \leq s$.
\end{enumerate}

Este algoritmo tiene como objetivo expresar el exponente $b$ de la forma $t \cdot s - r$ para un $s$. Para poder comprobar todas las posibilidades necesitamos que $s^2 \geq p - 1$, por lo tanto el $s$ a utilizar será mayor o igual a $\sqrt(p)$.

Para encontrar $s$ utilizaremos el algoritmo babilónico para calcular la raiz cuadrada y le sumaremos uno:

\begin{lstlisting}[caption={Algoritmo babilónico}]
INPUT: n
OUTPUT: resultado <- raiz cuadrada entera de n

resultado <- n
y <- 1

Mientras resultado > y:
	resultado <- (resultado + y) / 2
	y <- n / resultado

Devolver resultado
\end{lstlisting}

Una vez tenemos $s$, el algoritmo construye dos tablas, $l$ y $L$, donde:

\begin{itemize}
	\item $L = \{a^{s}, a^{2 \cdot s}, ..., a^{t \cdot s}, ..., a^{s \cdot s}\}$ donde a cada elemento le corresponde la secuencia de $t$, que es el rango $1 \leq t \leq s$.
	\item $l = \{c \cdot a^0, c \cdot a^1, ..., c \cdot a^{r} ..., c \cdot a^{s - 1} \}$ donde a cada elemento le corresponde la secuencia de $r$, que es el rango $0 \leq r \leq s - 1$.
\end{itemize}

Como vemos, en estas dos tablas $L$ daría un paso gigante y $l$ un paso enano (de ahí el nombre del algoritmo), de forma que si $L$ y $l$ se intersecan en, podremos conocer el logaritmo discreto en base $a$ de $c$ módulo $p$, es decir, $b$, con los valores de $s$, $r$ y $t$, ya $b = t \cdot s - r$.

Un detalle de cara a tener en cuenta en la implementación es que no es necesario calcular cada valor de la lista de forma independiente, para obtener un valor de $L$ basta con coger el elemento anterior y multiplicarlo por $a^s$, mientras que para obtener un valor de $l$ podemos coger el valor anterior y multiplicarlo por $a$.

Por lo tanto el algoritmo sería el siguiente:

\begin{lstlisting}[caption={Algoritmo paso enano paso gigante}]
INPUT: a, c, p
OUTPUT: resultado <- logaritmo discreto en base a de c modulo p
OUTPUT: -1 si p no es primo

resultado <- -1

Si p es primo:
	s <- raiz cuadrada de modulo - 1
	s <- s + 1
	l <- tabla de dos filas
	L <- tabla de dos filas

	valor_l <- c
	valor_L <- a^s mod p
	a_elevado_s <- valor_L

	Inserto en l la columna <valor_l, 0>
	Inserto en L la columna <valor_L, 1>

	encontrado_l <- busco si valor_L esta en l
	encontrado_L <- busco si valor_l esta en L

	i <- 1

	Mientras i <= s - 1 Y !encontrado_L Y !encontrado_l:
		valor_l <- (a * valor_l) mod p
		valor_L <- (a_elevado_s * valor_L) mod p

		Inserto en l la columna <valor_l, i>
		Inserto en L la columna <valor_L, i + 1>

		encontrado_l <- busco si valor_L esta en l
		encontrado_L <- busco si valor_l esta en L

		i <- i + 1

	Si encontrado_l:
		t <- i
		r <- 2a fila de la columna donde he encontrado L en l
		resultado <- t * s - r

	Si encontrado_L:
		t <- 2a fila de la columna donde he encontrado l en L
		r <- i - 1
		resultado <- t * s - r


Devolver resultado
\end{lstlisting}

Y con esto tenemos implementado el algoritmo del logaritmo discreto en $\mathbb{Z}_p$.

Ejemplo de ejecución:

\begin{lstlisting}
> ./bin/ejercicio5 2342352353346  325 123456789
AVISO: El numero primo dado no es primo, utilizando el siguiente primo:
	123456791
2342352353346 ^ 102554656 = 325 en Z_123456791
Comprobacion: 325

> ./bin/ejercicio5 2342352353346  124325 123456789
AVISO: El numero primo dado no es primo, utilizando el siguiente primo:
	123456791
2342352353346 ^ 23539859 = 124325 en Z_123456791
Comprobacion: 124325

> ./bin/ejercicio5 234235235233346  124325 123456789
AVISO: El numero primo dado no es primo, utilizando el siguiente primo:
	123456791
234235235233346 ^ 70188141 = 124325 en Z_123456791
Comprobacion: 124325
\end{lstlisting}


\section{Raices cuadradas modulares}

En esta sección se nos pide, dado un $n = p \cdot q$ siendo $p$ y $q$ dos números primos, implementar algoritmos para obtener un $r$ tal que $r^2 \equiv a \mod p$, siendo $a$ residuo cuadrático de $p$, es decir, su símbolo de Jacobi vale uno, y por otro lado obtener no solo una raiz, si no todas.

Para ambos algoritmos será necesario implementar el símbolo de Jacobi. El símbolo de Jacobi es una generalización del símbolo de Legendre, que resuelve el problema de que el termino inferior del símbolo no sea primo, demostrando que simplemente pierde algunas propiedas, aunque en nuestro caso, al trabajar con números primos no tendremos problema.

El símbolo de Jacobi dado un entero positivo $a$ y un entero positivo primo $p$ se define como:

\begin{itemize}
	\item $0$ si $a \equiv 0 \mod p$.
	\item $1$ si $a \not \equiv \mod p$ y para un entero $x$, $a \equiv x^2 \mod p$, por lo que es un residuo cuadrático.
	\item $1$ si $a \not \equiv \mod p$ y no existe un $x$, $a \equiv x^2 \mod p$.
\end{itemize}

De esta forma podemos implementar el símbolo de Jacobi de la siguiente forma:

\begin{lstlisting}[caption={Símbolo de Jacobi}]
INPUT: a, n
OUTPUT: resultado <- simbolo de jabobi de (a, n)

resultado <- -1

Si n > 0 Y n es impar:
	a <- a mod a
	t <- 1

	Mientras a != 0:
		Mientras a sea par
			a <- a / 2
			r <- n mod 8

			Si r == 3 o r == 5 :
				t <- -t

		Intercambio los valores de a y n

		Si a mod 4 == 3 Y n mod 4 == 3:
			t <- -t

		a <- a mod n

	Si n == 1:
		resultado <- t
	Si no:
		resultado <- 0

Devolver resultado
\end{lstlisting}

\subsection{Raiz cuadrada modular de a mod p}

Gracias al símbolo de Legendre podemos saber si un número $a$ tiene raiz cuadrada módulo $p$ (si el símbolo vale uno), ahora vamos a buscar dicha raiz:


\begin{lstlisting}[caption={Algoritmo para obtener la raiz cuadrada modular}]
INPUT: a, p <- numero primo
OUTPUT: resultado <- raiz cuadrada de a mod p

resultado <- -1

Si el simbolo de jacobi(a, p) == -1:
	n <- 2

	Mientras el simbolo de jacobi de (n, p) != -1:
		n <- n + 1

		Descomponemos p - 1 como 2^u \cdot s (almacenamos u y s)

		resultado <- a^{(s + 1) / 2} mod p
		b <- n^s mod p
		j <- 0

		Mientras j <= u - 2:
			Si (a^{-1} * resultado^2) ^{2^{u - 2 - j}} mod p == p - 1:
				resultado <- (resultado * b) mod p

			b <- b^2 mod p

			j <- j + 1

		// si queremos devolver la raiz mas pequena
		Si resultado > p / 2 :
			resultado <- p - resultado

Devolver resultado
\end{lstlisting}

Y con esto podemos obtener una raiz de $a$ en $\mathbb{Z}_p$.

Ejemplo de ejecución:

\begin{lstlisting}
> ./bin/ejercicio6_1 3323442 152212341157
Aviso: El numero introducido no es primo, utilizare el siguiente primo:
	152212341181
La raiz cuadrada de 3323442 modulo 152212341181 es : 42203554591
Comprobacion (utilizando powm): 42203554591^2 = 3323442


> ./bin/ejercicio6_1 3342 15227
La raiz cuadrada de 3342 modulo 15227 es : 7492
Comprobacion (utilizando powm): 7492^2 = 3342

> ./bin/ejercicio6_1 312312451342 152223713471473457
Aviso: El numero introducido no es primo, utilizare el siguiente primo:
	152223713471473463
La raiz cuadrada de 312312451342 modulo 152223713471473463 es :
	60165397748005818
Comprobacion (utilizando powm): 60165397748005818^2 = 312312451342
\end{lstlisting}



\subsection{Obtener todas las raices de a mod p*q}

Una vez tenemos un método para obtener una raiz cuadrada en $\mathbb{Z}_p$ con $p$ primo, para obtener todas raices de un $n$, sabiendo que $n = p \cdot q$, podemos obtener sus raices módulo $p$ y módulo $q$, y aplicando el Teorema Chino de los Restos obtener la raiz módulo $n$.

Si tenemos que:

$$x \equiv a \mod p$$
$$x \equiv b \mod q$$

A partir de la primera ecuación podemos obtener:

$$x = a + p \cdot k$$
$$a + p \cdot k \equiv b \mod q$$
$$p \cdot k \equiv b - a \mod q$$
$$k \equiv p^{-1} \cdot (b - a) \mod q$$
$$k = p^{-1} \cdot (b - a) + q \cdot t$$
$$x = a + p \cdot (p^{-1} \cdot (b - a) + q \cdot t )$$
$$x = a + p^{-1} \cdot (b - a) \cdot p + t \cdot p \cdot q$$

En nuestro caso solo queremos la solución particular, luego:
$$x = a + p^{-1} \cdot (b - a) \cdot p$$

Con esto podemos obtener un raiz, y para obtenerlas todas podemos aplicar el Teorema Chino de los Restos a las positbles combinaciones:

$$x \equiv -a \mod p$$
$$x \equiv b \mod q$$

$$x \equiv a \mod p$$
$$x \equiv -b \mod q$$


$$x \equiv -a \mod p$$
$$x \equiv -b \mod q$$

Pero en realidad, con calcular únicamente dos de ellas, las otras dos son sus opuestas, luego solo haría falta aplicarlo a dos parejas.

Con esto, la implementación sería:


\begin{lstlisting}[caption={Algoritmo raices cuadradas mod p*q}]
INPUT: a, p, q
OUTPUT: resultado <- con las raices de a mod p*q

resultado <- vacio

Si a es residuo cuadratico de p y si a es residuo cuadratico de q:
	n <- p * q

	raiz1_p <- raiz_cuadrada_mod(a, p)
	raiz2_p <- p - r1_p

	raiz1_q <- raiz_cuadrada_mod(a, q)

	solucion1 <- raiz1_p + p^{-1} * (raiz1_q - raiz1_p) * p
	solucion1 <- solucion1 mod n

	solucion2 <- raiz2_p + p^{-1} * (raiz1_q - raiz2_p) * p
	solucion2 <- solucion2 mod n

	solucion3 <- n - solucion1
	solucion4 <- n - solucion2

	Insertamos en resultado las cuatro soluciones sin permitir repetidos

Devolver resultado
\end{lstlisting}

Ejemplo de ejecución:

\begin{lstlisting}
> ./bin/ejercicio6_2 312312451342 152223713471473457 124
Aviso: El numero 152223713471473457 no es primo, utilizare el
	siguiente primo: 152223713471473463
Aviso: El numero 124  no es primo, utilizare el siguiente primo: 127
Las raices cuadradas de 312312451342 mod 19332411610877129801 son :
5115440860282091924
Comprobacion (usando powm):
	5115440860282091924^2 mod 19332411610877129801 = 312312451342

6637677994996826554
Comprobacion (usando powm):
	6637677994996826554^2 mod 19332411610877129801 = 312312451342

12694733615880303247
Comprobacion (usando powm):
	12694733615880303247^2 mod 19332411610877129801 = 312312451342

14216970750595037877
Comprobacion (usando powm):
	14216970750595037877^2 mod 19332411610877129801 = 312312451342

> ./bin/ejercicio6_2 4 3 5
Las raices cuadradas de 4 mod 15 son :
2        Comprobacion (usando powm): 2^2 mod 15 = 4
7        Comprobacion (usando powm): 7^2 mod 15 = 4
8        Comprobacion (usando powm): 8^2 mod 15 = 4
13       Comprobacion (usando powm): 13^2 mod 15 = 4
\end{lstlisting}

\section{Métodos de factorización}

En este ejercicio implementaremos dos formas de factorizar un número, primero utilizando el método de Fermat, y luego utilizando el algoritmo $\rho$ de Pollard.

\subsection{Factorización utilizando el método de Fermat}

La idea de este método es representar el número $n$ a factorizar como diferencia de dos cuadrados. Se basa en calcular la raiz cuadrada de $n$ a la que llamaremos $x$, y mientras $x^2 - n$ no sea un cuadrado perfecto incrementaremos en uno $x$.

De esta forma, cuando $x^2 - n$ sea un cuadrado perfecto, sabemos que $n = x \pm x^2 - n$.


\begin{lstlisting}[caption={Algoritmo raices cuadradas mod p*q}]
INPUT: n
OUTPUT: x, y <- pareja con la que podemos representar n como x +- y

x <- Raiz cuadrada entera de n

Si n != 1:
 	Hacer:
		x <- x + 1
		comprobacion <- x * x - n
		raiz_comprobacion <- raiz cuadrada entera de comprobacion
	Mientras raiz_comprobacion^2 != comprobacion

Si no:
	x <- 1
	raiz_comprobacion <- 0

Devolver x, raiz_comprobacion
\end{lstlisting}


Ejemplo de ejecución:

\begin{lstlisting}
> ./bin/ejercicio7_1 21
21 lo podemos expresar como (5 + 2) * (5 - 2)
Comprobacion: 21


> ./bin/ejercicio7_1 124125
124125 lo podemos expresar como (353 + 22) * (353 - 22)
Comprobacion: 124125

> ./bin/ejercicio7_1 124125213
124125213 lo podemos expresar como (11153 + 514) * (11153 - 514)
Comprobacion: 124125213
\end{lstlisting}


\subsection{Factorización utilizando el algoritmo de $\rho$ de Pollard}

Este método se basa en que dado un $n$ de la forma $n = p \cdot q$, si $p$ es un factor primo menor o igual que $\sqrt(n)$, si tengo una secuencia aleatoria de la forma $x_1, ..., x_k$, si $1 < mcd(|x_i - x_j|, n) < n$, entonces $p | mcd(|x_i - x_j|, n)$ y $n$ tendría un factor.

Para encontrar la cadena de aleatorios utilizaré una función de la forma $x^2 + c \mod n$ de forma que simule números aleatorios, comenzaremos con $c = 1$ y si no encontramos solución para esa $c$ la iremos incrementando, aunque eso no es parte del algoritmo, si no una forma de solucionar el problema de encontrar la secuancia de números aleatorios.

Para hacer más rápido el algoritmo haremos un paso lento y un paso rápido, de forma que calculemos a la vez un número y el siguiente.

La implementación sería la siguiente:


\begin{lstlisting}[caption={Algoritmo rho de Pollard}]
INPUT: n, c <- termino independiende de x^2 + c
OUTPUT: resultado <- factor no trivial de n

a <- 2
b <- 2
d <- 1

Mientras d == 1:
	a = a^2 + c mod n
	b = b^2 + c mod n
	b = b^2 + c mod n

	d = gcd(abs(a - b), n)

Si d == n:
	resultado <- FALLO
Si no:
	resultado <- d

Devolver resultado
\end{lstlisting}

De esta forma, si no encuentra un factor no trivial con $c$, simplemente podemos volver a lanzarlo con $c + 1$.

Ejemplo de ejecución:

\begin{lstlisting}
> ./bin/ejercicio7_2 1241252133416234234134617

Un factor no trivial de 1241252133416234234134617 es 3

> ./bin/ejercicio7_2 12412521334162342341123
Un factor no trivial de 12412521334162342341123 es 21

> ./bin/ejercicio7_2 1241252133416243
Un factor no trivial de 1241252133416243 es 13
\end{lstlisting}



\section{Comparación de tiempos}


% \begin{thebibliography}{9}
%
%
% \end{thebibliography}

\end{document}
